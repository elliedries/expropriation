Data first made publicly available in October 2012 (Version 1.0)


CORRECTIONS in Feburary 2013 (Version 1.1): 

(1) 2/22/13: Corrected Paraguay 1939-1948; it was 1940-1948 now it is 1939-1948; corrected start year, duration spell, and duration time

(2) 2/22/13: Corrected Greece 1948-1967; it was coded democracy from Jan 1 1948 to Jan 1 1967; now correctly coded democracy from Jan 1 1947 to Jan 1 1967. Updated Global case list and Global regimes. (h/t Kerim Can Kavakli)


CORRECTIONS in June 2014 (Version 1.2): 

(3) 6/20/14: Corrected Togo 1967-; it was coded democracy from Jan 1 1964 to Jan 1 1967; now correctly coded as autocratic from Jan 13, 1963 coup led by ex-Sgt Eyadema; the May 1963 election of Grunitzkey was not democratic. A New York Times article states: "With only a single slate of candidates yesterday, the result was a foregone conclusion."  ("Grunitzky Is Declared Winner in Togo's Election" New York Times, May 7 1963.)  First (1970, 207 ) writes: ``The army installed a civilian government, presided over by Olympia's political rival, Nicholas Grunitzky. Bodjolle made himself colonel, commander and chief-of-staff, while a former sergeant in the French army, Etienne Eyadema (who, it is widely believed, fired the shots that killed Olympio) became a major in an army expanded to 1,200 men. Two years later Eyadema ousted Bodjolle, to make himself colonel and commander, and subsequently to become Togo's head of state.'' (h/t Tyson Chandler)

(4) 6/20/14: Corrected Iran 1925-1979; previously the clean.do code and thus the GWFtscs.txt & GWFtscs.dta files had this regime incorrectly coded as gwf_party and gwf_monarch; only the latter is correct; all the GWFglobal data was correct (h/t Wonjun Song)

(5) 6/20/14: Corrected El Salvador 1932-1948; previously was military in GWFcases.dta and GWF Autocratic Regimes.xls but should be military-personal; this information was correct in the GWFglobal data (h/t Brett Ashley Leeds)

(6) 6/20/14: Corrected Cambodia 53-70 end date from March 17 to March 18, 1970; also corrected this date for the Cambodia 70-75 start date; GWFglobal data does not contain start dates so was not updated  (h/t John Chen) "Shinaouk Reported Out in a Coup By His Premier". New York Times, March 18, 1970 (page 1): "Prince Norodom Sihanouk... was overthrown today in his absence, the Pnompenh radio announced."

(7) 6/20/14: JW checked that all data has the correct date for end of the Mauritania 60-78 regime: it is July 10, 1978.  There do not appear to be any instances where this date was incorrectly entered as October 7, 1978. (h/t John Chen) 

Additional corrections since June 2014:

(8) 4/21/15: JW corrected Belarus 1993 -- which was listed as gwf_regime==party but gwf_party==0 -- changed this to gwf_party==1. This information was correct for all other files.  (h/t Xiao Yu)

(9) 5/3/16: The code book narrative of the regime end event for Pakistan (1958--71) has been corrected to add information on the type of the regime collapse event.  The original description is as follows:

12/20/71  Yahya Khan turned power over to Bhutto, whose party had won a plurality in West Pakistan - which after December 1971 was all that remained of Pakistan - in the December 1970 parliamentary elections.  These universal suffrage, direct elections were considered fair and expected to be transitional, but the Assembly had not been allowed to meet because the Awami League representing East Pakistani aspirations for greater autonomy had won a majority (Mook 1974, 110-11; Shehab 1995, 272-87).. Bhutto called the previously elected Assembly into session in spring 1972, and civilian government resumed (Middle East Journal 1972).

The corrected description is the following: 

12/20/71  In November 1969, Yahya announced parliamentary elections to return power to elected civilians and they were held in December 1970, but the Assembly did not meet because Bhutto and a faction of the military were unwilling to allow the Awami League, which had won, to take office (Feit 1973, 83; Mook 1974, 100, 108-9; Shehab 1995, 268-72).  This impasse led to violent demonstrations in East Pakistan, which the army attempted to put down amid great bloodshed. The flood of refugees into India caused the Indian army to intervene.  In December 1971 Yahya resigned in response to demonstrations after the military’s defeat by Indian forces in what was to become Bangladesh (Feit 1973, 83-84; Mook 1974, 110-11; Shehab 1995, 272-74); and Yahya Khan turned power over to Bhutto, whose party had won a plurality in West Pakistan in the December 1970 parliamentary elections.  These universal suffrage, direct elections were considered fair and expected to be transitional, but the Assembly had not been allowed to meet because the Awami League representing East Pakistani aspirations for greater autonomy had won a majority (Mook 1974, 110-111; Shehab 1995, 272-87).  Bhutto called the previously elected Assembly into session in spring 1972, and civilian government was resumed (Middle East Journal 1972). We code Yahya Khan's December 1971 resignation, which was precipitated by mass demonstrations, as the regime failure event.


